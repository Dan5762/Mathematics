\documentclass{article}
\usepackage[utf8]{inputenc}
\usepackage[a4paper, total={6in, 8in}]{geometry}
\usepackage{amsmath}

\title{Matrices and Vector Spaces}
\author{Daniel Long}

\begin{document}

\maketitle
  
\tableofcontents

\section{Vector Spaces}

\subsection{Conditions for a Vector Space}

A set of vectors \textbf{a},\textbf{b},\textbf{c},... form a linear vector space if:

\renewcommand{\theenumi}{\roman{enumi}}
\begin{enumerate}
\item Set is closed under commutative and associative addition
\item Set is closed under distributative and associative multiplication by a scalar
\item Set contains a null vector
\item Set contains a unity vector
\item A corresponding negative exists for all vectors
\end{enumerate}

\subsection{Vector Space Definitions}

\begin{description}
  \item[Span of a Set of Vectors] \hfill \\ The set of all vectors that can be expressed as a linear sum of the original set
  \item[Linear Dependence] \hfill \\ A set of vectors are linearly dependent if there exists a linear combination equal to the null vector
  \item[Basis Vectors] \hfill \\ In an N-dimensional vector space V any set of N linearly independent vectors form a set of basis vectors for V
\end{description}

\section{Inner Product}

\subsection{Properties of the Inner Product}

Using an orthonormal basis $ \hat{\textbf{e}} $ we can find the inner product as:
\begin{align}
	\langle \textbf{a} | \textbf{b} \rangle 	& = \langle a_1 \hat{\textbf{e}}_1 + a_2 \hat{\textbf{e}}_2 + ... + a_N \hat{\textbf{e}}_N |  b_1 \hat{\textbf{e}}_1 + b_2 \hat{\textbf{e}}_2 + ... + b_N \hat{\textbf{e}}_N \rangle \\
								& = \Sigma_{i=1}^{N} a_{i}^{*} b_{i} \langle \hat{\textbf{e}}_i | \hat{\textbf{e}}_j \rangle + \Sigma_{i=1}^{N} \Sigma_{j \neq i}^{N} a_{i}^{*} b_{i} \langle \hat{\textbf{e}}_i | \hat{\textbf{e}}_j \rangle \\
								& = \Sigma_{i=1}^{N} a_{i}^{*} b_{i}
\end{align}

\subsection{Inner Product Definitions}

\begin{description}
  \item[Orthogonality] \hfill \\ Vectors are orthogonal if $ \langle \textbf{a} | \textbf{b} \rangle  = 0 $
  \item[Vector Norm] \hfill \\ The norm of a vector is given by $ | \textbf{a} | = \langle \textbf{a} | \textbf{a} \rangle ^{\frac{1}{2}} $
  \item[Orthonormality] \hfill \\ A basis is orthonormal is $ \langle \hat{\textbf{e}}_i | \hat{\textbf{e}}_j \rangle  = \delta_{i,j} $
\end{description}

\section{Linear Operators}

In a vector space V with basis vectors $ \hat{\textbf{e}}_i $ we can define the action of the operator A as:
\begin{align}
	A \hat{\textbf{e}}_j = \Sigma_{i=1}^N A_{ij} \hat{\textbf{e}}_i
\end{align}

where $ A_{ij} $ is the ith component of the vector $ A \hat{\textbf{e}}_j $ in this basis and collectively the numbers $ A_{ij} $ are known as the components of the linear operator in the $ \hat{\textbf{e}}_i $ basis.

\subsection{Properties of the Linear Operators}

\begin{description}
  \item[Distributivity] \hfill \\ (A + B) \textbf{x} = A \textbf{x} + B \textbf{x}
  \item[Associativity with a scalar] \hfill \\ $ (\lambda A) \textbf{x} = \lambda (A \textbf{x}) $
  \item[Associativity with another linear operator] \hfill \\ $ (AB) \textbf{x} = A (B \textbf{x}) $
\end{description}

If a linear operator is singular it does not have an inverse

\section{Matrices}

If a linear operator A transforms vectors from an N dimensional vector space into a vector in an M dimensional vector space, then the operator A can be represented by the matrix:
\begin{align}
	A = \begin{pmatrix}
		A_{11} & A_{12} & ... & A_{1N} \\
		A_{21} & A_{22} & ... & A_{2N} \\
		... & ... & ... & ... \\
		 A_{M1} & A_{M2} & ... & A_{MN} \\
	\end{pmatrix}
\end{align}

\subsection{Basic Matrix Algebra}

Using the properties of linear operators we can show that matrices obey:
\begin{align}
	(A + B)_{ij} & = A_{ij} + B_{ij} \\
	(\lambda A)_{ij} & = \lambda A_{ij} \\
	(AB)_{ij} & = \Sigma_k A_{ik} B_{kj}
\end{align}

The transpose of a matrix product is given by:
\begin{align}
	(AB)^T = B^T A^T
\end{align}

as proven by:
\begin{align}
	(AB)_{ij}^T = (AB)_{ji} = \Sigma_k A_{jk} B_{ki} = \Sigma_k (B^T)_{ik} (A^T)_{kj} = (B^T A^T)_{ij}
\end{align}

\subsection{Trace of a Matrix}

The trace of a matrix is denoted by:
\begin{align}
	Tr(A) = \Sigma_{i=1}^N A_{ii}
\end{align}

The trace of a product of two matrices is independent of the order, as seen by:
\begin{align}
	Tr(AB) = \Sigma_{i=1}^N (AB)_{ii} = \Sigma_{i=1}^N \Sigma_{j=1}^N A_{ij} B_{ji} = \Sigma_{i=1}^N \Sigma_{j=1}^N B_{ji} A_{ij} = \Sigma_{j=1}^N (BA)_{jj} = Tr(BA)
\end{align}

\end{document}
